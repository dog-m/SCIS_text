%%%%%%%%%%%%%%%%%%%%%%%%%%%%%%%%%%%%%%%%%%%%%%%%%%%%%%%%%%%%%%%%%%%%%%%%%%%%%%%%
\intro
%%%%%%%%%%%%%%%%%%%%%%%%%%%%%%%%%%%%%%%%%%%%%%%%%%%%%%%%%%%%%%%%%%%%%%%%%%%%%%%%

По мере того как современные приложения становятся больше и сложнее, создание инструментов для управления этими программами становится все более трудным. В то же время потребность в инструментах для управления сложностью приложений растет. Все чаще требуются более продвинутые инструменты сбора информации для анализа, инструментирования и сбора трассировок, чтобы облегчить сложность разработки программного обеспечения, тестирования, отладки и моделирования. В то же время необходимы инструменты, которые изменяют программы для оптимизации, перевода, обеспечения совместимости и т. д.

Целью данной работы является ознакомление с процессом инструментирования и обзор некоторых существующих средств, позволяющих выполнять автоматизированное инструментирование программного обеспечения.

