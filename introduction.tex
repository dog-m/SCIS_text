%%%%%%%%%%%%%%%%%%%%%%%%%%%%%%%%%%%%%%%%%%%%%%%%%%%%%%%%%%%%%%%%%%%%%%%%%%%%%%%%
\intro
%%%%%%%%%%%%%%%%%%%%%%%%%%%%%%%%%%%%%%%%%%%%%%%%%%%%%%%%%%%%%%%%%%%%%%%%%%%%%%%%

По мере того как современные приложения становятся больше как по объему кодовой базы, так и по числу одновременно применяемых технологий, а вместе с тем, языков программирования, создание инструментов для управления этими программами становится все более трудным.
В то же время растет потребность во все более продвинутых инструментах сбора информации для анализа, инструментирования и сбора трассировок, чтобы облегчить процесс разработки программного обеспечения, тестирования, отладки и моделирования.
В то же время необходимы инструменты, которые изменяют программы для оптимизации, перевода, обеспечения совместимости и т.д.

Вместе с тем, в последнее время произошел бурный рост и развитие множества различных языков программирования.
Как появление совершенно новых, в особенности -- предметно-ориентированных, так и расширение возможностей проверенных временем промышленных языков.
В связи с этим остро встает вопрос отладки и поиска ошибок в программном обеспечении, разработка которого ведется с использованием этих возможностей.

В соответствии с этим, объектом исследования в данной работе выступает процесс инструментирования исходного текста программ.
Предметом исследования же является оценка возможностей для построения инструмента, позволяющего выполнить этот процесс для программного обеспечения, которое было создано с применением различных языков программирования.

Целями данной работы являются проектирование, разработка и апробация прототипа мультиязычной расширяемой системы инструментирования исходных текстов программ, а именно -- генератора систем инструментирования.

Работа состоит из 5 разделов.
В первом разделе представлен обзор предметной области, рассмотрены основные виды инструментирования и некоторые существующие системы, позволяющие автоматизировать этот процесс.
Во втором разделе выполняется постановка сопутствующих задач, стоящих на пути создания автоматизированной системы инструментирования, а также рассматриваются требования и ограничения, стоящие при их выполнении вместе с возможными решениями.
Третий раздел посвящен проектированию системы инструментирования программного кода.
В четвертом разделе описана реализация предложенного подхода в виде прототипа генератора систем автоматизации инструментирования программ.
Тестированию разработанного инструмента в условиях работы с различными языками программирования посвящен пятый раздел.
