%%%%%%%%%%%%%%%%%%%%%%%%%%%%%%%%%%%%%%%%%%%%%%%%%%%%%%%%%%%%%%%%%%%%%%%%%%%%%%%%
\chapter{Постановка задачи и выбор средств}
%%%%%%%%%%%%%%%%%%%%%%%%%%%%%%%%%%%%%%%%%%%%%%%%%%%%%%%%%%%%%%%%%%%%%%%%%%%%%%%%

***

%%%%%%%%%%%%%%%%%%%%%%%%%%%%%%%%%%%%%%%%%%%%%%%%%%%%%%%%%%%%%%%%%%%%%%%%%%%%%%%%
\section{Детальная постановка задачи}
%%%%%%%%%%%%%%%%%%%%%%%%%%%%%%%%%%%%%%%%%%%%%%%%%%%%%%%%%%%%%%%%%%%%%%%%%%%%%%%%

=== ЧЕРНОВОЙ ВАРИАНТ ===

Необходимо разработать мультиязычную расширяемую программную систему, позволяющую производить автоматическое инструментирование исходного текста программы в соответствии с правилами, задаваемыми пользователем такой системы.

Следует принять во внимание возможность пользователя выполнять инструментирование исходных текстов синтаксически-корректной программы, созданной с использованием любого языка программирования, грамматику которого может описать пользователь формально с использованием языка TXL.

Особое внимание следует уделить возможности инструментирования основных управляющих конструкций выбранного пользователем языка программирования.

=== ЧЕРНОВОЙ ВАРИАНТ ===

***

%%%%%%%%%%%%%%%%%%%%%%%%%%%%%%%%%%%%%%%%%%%%%%%%%%%%%%%%%%%%%%%%%%%%%%%%%%%%%%%%
\section{??? Выбор пути (глобально)}
%%%%%%%%%%%%%%%%%%%%%%%%%%%%%%%%%%%%%%%%%%%%%%%%%%%%%%%%%%%%%%%%%%%%%%%%%%%%%%%%

***

Инструментирование исходного текста.
Только добавление текста -- вцелом порядок может быть такой: удаление, обновление, добавление.
Правила инструментирования задаются декларативно.
Зависимые правила?

***

%%%%%%%%%%%%%%%%%%%%%%%%%%%%%%%%%%%%%%%%%%%%%%%%%%%%%%%%%%%%%%%%%%%%%%%%%%%%%%%%
\section{Средства разработки}
%%%%%%%%%%%%%%%%%%%%%%%%%%%%%%%%%%%%%%%%%%%%%%%%%%%%%%%%%%%%%%%%%%%%%%%%%%%%%%%%

***
TXL, C++, Boost library, argparse, tinyxml2

%%%%%%%%%%%%%%%%%%%%%%%%%%%%%%%%%%%%%%%%%%%%%%%%%%%%%%%%%%%%%%%%%%%%%%%%%%%%%%%%
\section{Выводы}
%%%%%%%%%%%%%%%%%%%%%%%%%%%%%%%%%%%%%%%%%%%%%%%%%%%%%%%%%%%%%%%%%%%%%%%%%%%%%%%%

***
