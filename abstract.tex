
\keywords{%
  инструментирование,
  txl,
  трансформация программного кода,
  синтез программного кода,
  предметно-ориентированный язык
}

\abstractcontent{
Тема выпускной квалификационной работы: <<Генератор систем автоматизации инструментирования программ>>.

Данная работа посвящена исследованию процесса инструментирования исходного текста программ и оценке возможностей для построения инструмента, позволяющего выполнить этот процесс для программного обеспечения, которое было создано с применением различных языков программирования.

В данной работе был разработан подход к решению задачи инструменирования исходного текста, основывающийся на нисходящем однопроходном методе в условиях полной доступности информации для определения рабочего контекста, задаваемого конечным пользователем.
Разработанный подход был реализован в виде прототипа генератора систем инструменирования, который далее был протестирован на нескольких языках, придерживающихся парадигме структурного программирования.

Научная новизна данной работы заключается в рассмотрении и анализе возможности построения расширяемой многоязычной системы инструментирования исходных текстов программ, созданных с использованием различных языков программирования.

Результаты проведенной работы могут быть использованы при практической реализации универсальной многоязычной автоматизированной системы инструменирования.

}

\keywordsen{
  instrumentation,
  txl,
  source code transformation,
  source code synthesis,
  dsl
}

\abstractcontenten{
The subject of the graduate qualification work is ``Generator of automation systems for software instrumentation''.

This work is devoted to the study of the process of instrumenting the source code of programs and evaluating the possibilities for building a tool that allows user to perform such process for software that was created using various programming languages.

In this work, an approach was developed to solve the problem of instrumentation of the source text, based on a top-down single-pass method in conditions of complete availability of information to determine the working context specified by the end user.
The developed approach was implemented as a prototype of an instrumentation system generator, which was further tested in several languages with structured programming paradigm.

The scientific novelty of this work is to consider and analyze the possibility of building an extensible multi-language system for instrumentation of source texts of programs created using various programming languages.

The results of this work can be used in the practical implementation of a universal multilingual automated instrumentation system.
}
