%%%%%%%%%%%%%%%%%%%%%%%%%%%%%%%%%%%%%%%%%%%%%%%%%%%%%%%%%%%%%%%%%%%%%%%%%%%%%%%%
\conclusion
%%%%%%%%%%%%%%%%%%%%%%%%%%%%%%%%%%%%%%%%%%%%%%%%%%%%%%%%%%%%%%%%%%%%%%%%%%%%%%%%

В данной работе была рассмотрена задача инструментирования исходных текстов программ.
Был выполнен обзор некоторых существующих систем, позволяющих решать эту задачу автоматизированно, при этом был выявлен их общий недостаток -- сильная ограниченность одним или несколькими яыками программирования, без возможности расширения функциональности для каких-либо других или появляющихся новых языков.

В ходе данной работы был предложен подход к автоматизированному синтезу систем инструментирования исходных текстов программ, основанный на однократном проходе по синтаксическому дереву разбора с применением определяемых пользователем контекстов.
В данном случае рассматривалась ситуация, в которой для определения контекста доступна вся необходимая информации.
Кроме того, были представлены предпосылки к разработке альтернативного подхода к решению поставленной задачи.

Вместе с тем, был рассмотрен процесс проектирования и разработки мультиязычной расширяемой системы, применяющей этот подход.

Кроме того, результатом данной работы является реализованный прототип генератора систем автоматизации инструментирования исходных текстов программ.
Тестирование разработанного прототипа было выполнено как на простых тестовых приложениях, так и на кодовой базе проектов промышленного уровня, в обоих случаях продемонстрировав свою применимость.

Дальнейшее развитие проекта возможно в направлении рассмотрения и реализации альтернативных или дополняющих подходов к задаче инструментирования, а также оценки и манипулирования рабочими контекстами инструментирования.
