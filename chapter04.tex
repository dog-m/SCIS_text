%%%%%%%%%%%%%%%%%%%%%%%%%%%%%%%%%%%%%%%%%%%%%%%%%%%%%%%%%%%%%%%%%%%%%%%%%%%%%%%%
\chapter{Реализация}
%%%%%%%%%%%%%%%%%%%%%%%%%%%%%%%%%%%%%%%%%%%%%%%%%%%%%%%%%%%%%%%%%%%%%%%%%%%%%%%%

Программная реализация.

%%%%%%%%%%%%%%%%%%%%%%%%%%%%%%%%%%%%%%%%%%%%%%%%%%%%%%%%%%%%%%%%%%%%%%%%%%%%%%%%
\section{Синтез цепочек TXL функций}
%%%%%%%%%%%%%%%%%%%%%%%%%%%%%%%%%%%%%%%%%%%%%%%%%%%%%%%%%%%%%%%%%%%%%%%%%%%%%%%%

Поскольку язык, который использует утилита TXL принадлежит к семейству функциональных языков программирования, это ограничение заставляет передавать собираемую информацию посредством параметров, с которыми вызываются последующие функции в цепочке.

***

Цепочки вызываются из функции `main'.

***

Ниже представлена общая структура цепочки вызовов:
\begin{enumerate}
  \item C-функции -- функции сбора информации.
  \item F-функции -- функции фильтрации.
  \item R-функций -- функций уточнения контекста.
  \item I-функции -- функции инструментирования.
\end{enumerate}

Кроме этого необходимы следующие вспомогательные TXL функции:
\begin{itemize}
  \item G-функции -- функции получения узлов, содержащих полезные значения (так называемых ''точек интереса''), из промежуточных узлов дерева разбора, имеющих TXL тип, который описывает какую-либо важную синтаксическую конструкцию целевого языка программирования.
  \item H-функции -- функции проверки принадлежности определенному контексту по передаваемым параметрам.
  \item W-функции -- функции-обертки стандартных операторов сравнения и поиска, выполняемых над текстовыми данными.
\end{itemize}

***

%%%%%%%%%%%%%%%%%%%%%%%
\subsection{Функции сбора информации}
%%%%%%%%%%%%%%%%%%%%%%%

***

C-функции -- это TXL функции, предназначенные для сбора информации (узлов древа разбора), которая используется для определения, соответствует-ли рассматриваемый/обрабатываемый узел дерева разбора выбранному контексту инструментирования.

Ниже приведен пример C-функции:
\begin{lstlisting}[frame=single, language=TXL]
rule CollectorFuncB kw_Class [class_declaration]
  skipping [method_declaration]
  replace $ [method_declaration]
    __NODE__ [method_declaration]
  by
    __NODE__
      [CollectorFuncC kw_Class __NODE__]
end rule
\end{lstlisting}

***

%%%%%%%%%%%%%%%%%%%%%%%
\subsection{Функции фильтрации по контексту}
%%%%%%%%%%%%%%%%%%%%%%%

***

F-функции -- это TXL функции, предназначенные для фильтрации.

Ниже приведен пример F-функции:
\begin{lstlisting}[language=TXL]
function filteringFunction
    kw_Class [class_declaration] kw_Method [class_declaration]
  skipping [method_declaration]
  replace $ [method_declaration]
    __NODE__ [method_declaration]
  construct __VOID__ [any]
    % void
  where
    __VOID__ [belongs_to_class_and_method kw_Class kw_Method]
  by
    __NODE__
      [refinerFunction kw_Class kw_Method]
end function
\end{lstlisting}

***

%%%%%%%%%%%%%%%%%%%%%%%
\subsection{Функции уточнения контекста}
%%%%%%%%%%%%%%%%%%%%%%%

После проверки контекста функцией-фильтром, производится вызов первой функции из последовательности R-функций.

R-функции -- это TXL функции, предназначенные для ...

Ниже приведен пример R-функции:
\begin{lstlisting}[frame=single, language=TXL]
function refinerFunction kw_Class [class_declaration] kw_Method [class_declaration]
  skipping ???
  replace * [repeat declaration_or_statement]
    __NODE__ [if_statement] __TAIL__ [repeat declaration_or_statement]
  construct __SINGLE_BOX_ARRAY__ [repeat declaration_or_statement]
    __NODE__ % +empty
  construct __PROCESSED__ [repeat declaration_or_statement]
    __SINGLE_BOX_ARRAY__ [instrumentationFunction kw_Class kw_Method]
  by
    __PROCESSED__ [. __TAIL__]
end function
\end{lstlisting}

***

%%%%%%%%%%%%%%%%%%%%%%%
\subsection{Функции инструментирования}
%%%%%%%%%%%%%%%%%%%%%%%

***

I-функции -- это TXL функции, предназначенные для...

Ниже приведен пример I-функции:
\begin{lstlisting}[frame=single, language=TXL]
function instrumentationFunction kw_Class [class_declaration] kw_Method [class_declaration]
  replace [class_or_interface_body]
    __NODE__ [class_or_interface_body]
  deconstruct __NODE__
    '{ RepeatClassBodyDeclaration [repeat class_body_declaration] '} OptLiteral [opt ';]
  construct FILE [stringlit]
    _ [+ "*dir/test.java*"]
  construct POINTCUT [id]
    'all
  construct NODE [id]
    _ [typeof __NODE__]
  construct msg [stringlit]
    _  [quote POINTCUT] [+ " "] [quote NODE] [+ " block in "] [quote FILE] [+ " class, in "] [quote METHOD_NAME] [+ " method"]
  by
    'iLogger.log(Level.FINE, msg ');
    'if '( Condition ') '{ Statement '} OptElseClause
end function
\end{lstlisting}

***

%%%%%%%%%%%%%%%%%%%%%%%
\subsection{Вспомогательные TXL функции}
%%%%%%%%%%%%%%%%%%%%%%%

***

%%%%%%%%%%%
\subsubsection{H-функции}
%%%%%%%%%%%

***

Ниже приведен пример H-функций:
\begin{lstlisting}[frame=single, language=TXL]
function __belongs_to_context___namespace_m kw_Method [procedure_impl_decl]
	match [any]
		_ [any]
	construct __VOID__ [any]
		% void
	construct POI_METHOD_NAMESPACE_str [stringlit]
		_ [__POI_get___POI_METHOD_NAMESPACE kw_Method]
	where
		__VOID__ [__std__equal POI_METHOD_NAMESPACE_str "Main."]
end function


function __not__belongs_to_context___namespace_m kw_Method [procedure_impl_decl]
	match [any]
		_ [any]
	construct __VOID__ [any]
		% void
	where not
		__VOID__ [__belongs_to_context___namespace_m kw_Method]
end function
\end{lstlisting}

***

%%%%%%%%%%%
\subsubsection{G-функции}
%%%%%%%%%%%

***

Ниже приведен пример G-функции:
\begin{lstlisting}[frame=single, language=TXL]
function __POI_get___POI_METHOD_NAME_FULL kw_Method [procedure_impl_decl]
	replace [stringlit]
		_ [stringlit]
	deconstruct kw_Method
		    ProcedureIntfDecl0 [procedure_intf_decl] _ [nested_decl_block] _ [procedure_body_semi]
	deconstruct ProcedureIntfDecl0
		    ProcedureSignature1 [procedure_signature] _ [repeat semi_directive] _ [opt ';]
	deconstruct ProcedureSignature1
		    _ [opt 'class] _ [procedure_keyword] ProcedureId2 [opt procedure_id] _ [opt formal_parameters] _ [opt colon_type]
	construct ProcedureId2_str [stringlit]
		_ [quote ProcedureId2]
	by
		ProcedureId2_str
end function
\end{lstlisting}

***

%%%%%%%%%%%
\subsubsection{W-функции}
%%%%%%%%%%%

***

Ниже приведен пример W-функции:
\begin{lstlisting}[frame=single, language=TXL]
function __std__lower_equal A [stringlit] B [stringlit]
	match [any]
		_ [any]
	where
		A [<= B]
end function
\end{lstlisting}

***

%%%%%%%%%%%%%%%%%%%%%%%
\subsection{???}
%%%%%%%%%%%%%%%%%%%%%%%

***

%%%%%%%%%%%%%%%%%%%%%%%%%%%%%%%%%%%%%%%%%%%%%%%%%%%%%%%%%%%%%%%%%%%%%%%%%%%%%%%%
\section{Выводы}
%%%%%%%%%%%%%%%%%%%%%%%%%%%%%%%%%%%%%%%%%%%%%%%%%%%%%%%%%%%%%%%%%%%%%%%%%%%%%%%%

Текст.
