%%%%%%%%%%%%%%%%%%%%%%%%%%%%%%%%%%%%%%%%%%%%%%%%%%%%%%%%%%%%%%%%%%%%%%%%%%%%%%%%
\chapter{Разработка}
%%%%%%%%%%%%%%%%%%%%%%%%%%%%%%%%%%%%%%%%%%%%%%%%%%%%%%%%%%%%%%%%%%%%%%%%%%%%%%%%

Креатив.

%%%%%%%%%%%%%%%%%%%%%%%%%%%%%%%%%%%%%%%%%%%%%%%%%%%%%%%%%%%%%%%%%%%%%%%%%%%%%%%%
\section{Синтез TXL функций}
%%%%%%%%%%%%%%%%%%%%%%%%%%%%%%%%%%%%%%%%%%%%%%%%%%%%%%%%%%%%%%%%%%%%%%%%%%%%%%%%

Основная идея нисходящего подхода заключается в постепенном спуске по дереву разбора, которое генерирует внутри себя утилита TXL, от корневого узла к листьям, сохраняя по пути следования информацию для определения контекста, в котором ниходятся обрабатываемые узлы и выполняется инструментирование.

Поскольку язык, который использует утилита TXL принадлежит к семейству функциональных языков программирования, это ограничение заставляет передавать собираемую информацию посредством параметров, с которыми вызываются последующие функции в цепочке.

???

%%%%%%%%%%%%%%%%%%%%%%%
\subsection{Общая структура цепочки вызовов}
%%%%%%%%%%%%%%%%%%%%%%%

\begin{enumerate}
  \item C-функции -- функции сбора информации.
  \item F-функции -- функции фильтрации.
  \item R-функций -- функций уточнения контекста.
  \item I-функции -- функции инструментирования.
\end{enumerate}

Кроме этого необходимо существование:
\begin{itemize}
  \item G-функции -- функции получения узлов, содержащих полезные значения (так называемых ''точек интереса''), из промежуточных узлов дерева разбора, имеющих TXL тип, который описывает какую-либо важную синтаксическую конструкцию целевого языка программирования.
  \item H-функции -- функции проверки принадлежности определенному контексту по передаваемым параметрам.
\end{itemize}

%%%%%%%%%%%%%%%%%%%%%%%
\subsection{Функции сбора информации}
%%%%%%%%%%%%%%%%%%%%%%%

C-функции -- функции сбора информации.

%%%%%%%%%%%%%%%%%%%%%%%
\subsection{Функция фильтрации}
%%%%%%%%%%%%%%%%%%%%%%%

F-функции -- функции фильтрации.

%%%%%%%%%%%%%%%%%%%%%%%
\subsection{Функции уточнения контекста}
%%%%%%%%%%%%%%%%%%%%%%%

После проверки контекста функцией-фильтром, производится вызов первой функции из последовательности R-функций -- функций уточнения контекста.

%%%%%%%%%%%%%%%%%%%%%%%
\subsection{Функция инструментирования}
%%%%%%%%%%%%%%%%%%%%%%%

I-функции -- функции инструментирования.

%%%%%%%%%%%%%%%%%%%%%%%%%%%%%%%%%%%%%%%%%%%%%%%%%%%%%%%%%%%%%%%%%%%%%%%%%%%%%%%%
\section{Выводы}
%%%%%%%%%%%%%%%%%%%%%%%%%%%%%%%%%%%%%%%%%%%%%%%%%%%%%%%%%%%%%%%%%%%%%%%%%%%%%%%%

Текст.
